%%% LaTeX Template: Two column article
%%%
%%% Source: http://www.howtotex.com/
%%% Feel free to distribute this template, but please keep to referal to http://www.howtotex.com/ here.
%%% Date: February 2011

%%% Preamble
\documentclass[	DIV=calc,%
							paper=a4,%
							fontsize=11pt]{scrartcl}	 					% KOMA-article class

\usepackage[spanish]{babel}										% English language/hyphenation
\usepackage[protrusion=true,expansion=true]{microtype}				% Better typography
\usepackage{amsmath,amsfonts,amsthm}					% Math packages
%\usepackage[pdftex]{graphicx}									% Enable pdflatex
\usepackage[svgnames]{xcolor}									% Enabling colors by their 'svgnames'
\usepackage[hang, small,labelfont=bf,up,textfont=it,up]{caption}	% Custom captions under/above floats
\usepackage{epstopdf}												% Converts .eps to .pdf
\usepackage{subfig}													% Subfigures
\usepackage{booktabs}												% Nicer tables
\usepackage{fix-cm}													% Custom fontsizes
\usepackage{hyperref}

\usepackage[T1]{fontenc}
\usepackage[utf8]{inputenc}
\usepackage[light,math]{kurier}


%%% Custom sectioning (sectsty package)
\usepackage{sectsty}													% Custom sectioning (see below)
\allsectionsfont{%															% Change font of al section commands
	\usefont{T1}{mdugm}{b}{it}%										% bch-b-n: CharterBT-Bold font
	}

\sectionfont{%																% Change font of \section command
	\usefont{T1}{mdugm}{b}{it}%										% bch-b-n: CharterBT-Bold font
	}

%%% Headers and footers
\usepackage{fancyhdr}												% Needed to define custom headers/footers
	\pagestyle{fancy}														% Enabling the custom headers/footers
\usepackage{lastpage}

% Header (empty)
\lhead{}
\chead{}
\rhead{}
% Footer (you may change this to your own needs)
\lfoot{\footnotesize \miit{Alejandro Alcalde Barros} \textbullet ~}
\cfoot{}
\rfoot{\footnotesize Página \thepage\ de \pageref{LastPage}}	% "Page 1 of 2"
\renewcommand{\headrulewidth}{0.0pt}
\renewcommand{\footrulewidth}{0.4pt}



%%% Creating an initial of the very first character of the content
\usepackage{lettrine}
\newcommand{\initial}[1]{%
     \lettrine[lines=3,lhang=0.3,nindent=0em]{
     				\color{DarkGoldenrod}
     				{\textsf{#1}}}{}}



%%% Title, author and date metadata
\usepackage{titling}															% For custom titles

\newcommand{\HorRule}{\color{DarkGoldenrod}%			% Creating a horizontal rule
									  	\rule{\linewidth}{1pt}%
										}

\pretitle{\vspace{-30pt} \begin{flushleft} \HorRule
				\fontsize{50}{50} \usefont{T1}{kurier}{l}{it} \color{DarkRed} \selectfont
				}
\title{Memoria Relación de Ejercicios PDDL}					% Title of your article goes here
\posttitle{\par\end{flushleft}\vskip 0.5em}

\preauthor{\begin{flushleft}
					\large \lineskip 0.5em \usefont{T1}{mdugm}{m}{it} \color{DarkRed}}
\author{Alejandro Alcalde, }											% Author name goes here
\postauthor{\footnotesize \usefont{OT1}{mdugm}{m}{it} \color{Black}
					Universidad de Granada 								% Institution of author
					\par\end{flushleft}\HorRule}

\date{\usefont{T1}{mdugm}{b}{it}\selectfont\today}																				% No date

\newcommand{\miit}[1]{{\usefont{T1}{mdugm}{m}{it}\selectfont #1}}

\usepackage{minted}

\newminted{bash}{
		numbersep=5pt,
		autogobble=true,
		frame=lines,
		framesep=2mm,
		fontsize=\scriptsize,
		tabsize=2,
		fontfamily=DejaVuSansMono-TLF,
}

\newminted{newlisp}{
		numbersep=5pt,
		autogobble=true,
		frame=lines,
		framesep=2mm,
		fontsize=\scriptsize,
		tabsize=2,
		fontfamily=DejaVuSansMono-TLF,
}

\newmintedfile[myLisp]{newlisp}{
    linenos,
    numbersep=5pt,
		autogobble=true,
    frame=lines,
    framesep=2mm,
    fontsize=\scriptsize,
		tabsize=2,
		fontfamily=DejaVuSansMono-TLF,
}

\newcommand{\lispscript}[2]{
    \myLisp[label=#2]{#1}
}

\newmintinline{newlisp}{fontsize=\scriptsize, fontfamily=DejaVuSansMono-TLF}

%%% Begin document
\begin{document}
\maketitle
\thispagestyle{fancy} 			% Enabling the custom headers/footers for the first page
% The first character should be within \initial{}
\initial{E}\textbf{n ésta memoria se describe cómo se han planteado, y resuelto, los distintos problemas propuestos sobre PDDL en la asignatura Técnicas de los Sisteas Inteligentes.}

\section{Ejercicio 1}

\miit{Escribir un dominio de planificación en PDDL para que un planificador pueda encontrar planes de actuación para uno o varios robots como
soluciones a problemas de distribución de paquetes entre habitaciones.}

\subsection{Representación de los objetos}

Para representar los objetos en éste dominio, se ha usado herencia. Para ello, se definió un \newlispinline/ojetofisico/, extendiendo de \newlispinline/object/. \newlispinline/cosas/ extiende de \newlispinline/objetofisico/, éstas \newlispinline/cosas/, serán los paquetes y los robots.

\subsection{Predicados}

Se han definido los siguientes predicados:

\begin{itemize}

	\item \newlispinline/(at ?r - objetofisico ?h - habitacion)/: Verdadero sii \newlispinline/?r/ es un \newlispinline/objetofisico/, \newlispinline/?h/ es una \newlispinline/habitacion/ y el objeto \newlispinline/?r/ está en la habitación \newlispinline/?h/. Para poder reutilzar éste predicado para indicar que, tanto un paquete, como un robot, están en una habitación, el tipo de objeto que se requiere es de tipo \newlispinline/objetofisico/, ya que \newlispinline/paquete, robot/ extienden de \newlispinline/cosas/, que a su vez extiende de \newlispinline/objetofisico/. Por tanto al predicado \newlispinline/at/ podremos pasarle como primer parámetro tanto un paquete como un robot.

	\item \newlispinline/(conectada ?h1 - habitacion ?h2 - habitacion)/: Verdadero sii \newlispinline/?h1/ y \newlispinline/?h2/ son \newlispinline/habitaciones/ y están conectadas.

	\item \newlispinline/(free ?r - robot)/: Verdadero sii \newlispinline/?r/ es un \newlispinline/robot/, y el robot \newlispinline/?r/ está libre, es decir, puede coger un paquete.

	\item \newlispinline/(carry ?p - paquete ?r - robot)/: Verdadero sii \newlispinline/?p/ es un \newlispinline/paquete/, \newlispinline/?r/ un \newlispinline/robot/ y el robot \newlispinline/?r/ lleva el \newlispinline/paquete ?p/.
\end{itemize}

\subsection{Acciones}

	Las acciones posibles en el dominio son:

	\begin{itemize}
		\item \newlispinline/move/: Representa la posibilidad de que el robot se mueva entre habitaciones.
		\begin{itemize}
			\item Parámetros: \newlispinline/(?r - robot ?from ?to - habitacion)/.
			\item Precondiciones: \newlispinline/(at ?r ?from),(conectada ?from ?to)/, que el robot \newlispinline/?r/ esté en la habitación de partida
			\newlispinline/?from/, y que ésta última esté conectada con la habitación \newlispinline/?to/.
			\item Efectos: \newlispinline/(at ?r ?to), (not (at ?r ?from))/, el robot \newlispinline/?r/ estará en la habitación \newlispinline/?to/, y dejará de estar en la habitación \newlispinline/?from/.
		\end{itemize}
		\item \newlispinline/pick/: Representa la posibilidad de que el robot coja un objeto de la habitación, en éste caso un paquete.
		\begin{itemize}
			\item Parámetros: \newlispinline/(?obj - paquete ?h - habitacion ?r - robot)/.
			\item Precondiciones: \newlispinline/(at ?obj ?h), (at ?r ?h), (free ?r)/. El objeto \newlispinline/?obj/ a coger está en la habitación
			 \newlispinline/?h/, al igual que el robot \newlispinline/?r/. El robot \newlispinline/?r/ está libre para coger el objeto.
			\item Efectos: \newlispinline/(carry ?obj ?r), (not (at ?obj ?h)), (not (free ?r))/. El robot \newlispinline/?r/ lleva el objeto \newlispinline/?obj/, el objeto ya no se encuentra en la habitación \newlispinline/?h/ y el robot \newlispinline/?r/ ya no está libre.
		\end{itemize}
		\item \newlispinline/drop/: Representa la posibilidad de que el robot deje un objeto en una habitación.
		\begin{itemize}
			\item Parámetros: \newlispinline/(?obj - paquete ?h - habitacion ?r - robot)/.
			\item Precondiciones: \newlispinline/(carry ?obj ?r), (at ?r ?h)/. El robot \newlispinline/?r/ lleva al objeto \newlispinline/?obj/ y el robot
			está en la habitación \newlispinline/?h/.
			\item Efectos: \newlispinline/(at ?obj ?h), (free ?r), (not (carry ?obj ?r))/. El el objeto \newlispinline/?obj/ está en la habitación \newlispinline/?h/, el robot \newlispinline/?r/ está libre y ya no lleva al objeto \newlispinline/?obj/.
		\end{itemize}
	\end{itemize}

A continuación se muestra el fichero con el código fuente.

\lispscript{../d1.pddl}{Dominio1.pddl}

\subsection{Problemas}

Para éste ejercicio se han diseñado dos tipos de problemas, el primero más sencillo, con un solo robot, dos paquetes y tres habitaciones. Las habitaciones conectadas son la primera con la segunda, y la segunda con la tercera. A continuación se muestra la definición del problema:

\lispscript{../p0e1.pddl}{p0e1.pddl}

La ejecución de FF muestra el siguiente plan:

\begin{bashcode}
	        0: PICK P1 HAB0 R1
	        1: MOVE R1 HAB0 HAB1
	        2: MOVE R1 HAB1 HAB2
	        3: DROP P1 HAB2 R1
	        4: PICK P2 HAB2 R1
	        5: MOVE R1 HAB2 HAB1
	        6: MOVE R1 HAB1 HAB0
	        7: DROP P2 HAB0 R1
	        8: MOVE R1 HAB0 HAB1
\end{bashcode}

El segundo problema tiene dos robots, dos paquetes y tres habitaciones, contectadas igual que el problema anterior.

\lispscript{../p1e1.pddl}{p1e1.pddl}

El plan calculado es:

\begin{bashcode}
	        0: PICK P2 HAB2 R2
	        1: PICK P1 HAB0 R1
	        2: MOVE R1 HAB0 HAB1
	        3: MOVE R2 HAB2 HAB1
	        4: MOVE R2 HAB1 HAB0
	        5: DROP P2 HAB0 R2
	        6: MOVE R2 HAB0 HAB1
	        7: MOVE R1 HAB1 HAB2
	        8: DROP P1 HAB2 R1
	        9: MOVE R1 HAB2 HAB1
\end{bashcode}

\section{Ejercicio 2}

\miit{Escribir un dominio de planificación en PDDL, modificando el dominio del anterior
ejercicio, de tal manera que se tenga en cuenta que la acción de moverse de una
habitación a otra consume una cantidad de batería y, por tanto, requiere que el robot
tenga nivel de batería para moverse. Además, considerar que hay una nueva acción de
carga de batería que permite reponer la batería. Considerar para ello que se ha
definido un predicado (cambio n1 n2 – nivelbat) que representa un cambio en el nivel
de batería desde un nivel n1 a un nivel n2. En En el material de esta sesión de prácticas
hay un fichero ejemplo de un problema para este tipo de dominio.}

\subsection{Dominio}
\label{sub:Dominio}

En éste dominio se ha definido un nuevo objeto \newlispinline/flevel/, que representará los niveles de batería del robot.

Además, se han añadido dos predicados nuevos, \newlispinline/cambio( ?n1 ?n2 - flevel)/ y \newlispinline/(batterylevel ?r - robot ?f - flevel)/. El primero realizará el cambio de una batería por otra. El segundo especifica el nivel de batería del robot \newlispinline/?r/.

En la acción \newlispinline/move/:

\begin{itemize}
	\item En los parámetros se pasan dos niveles de batería. Uno de ellos especifica el nivel de batería del robot antes de moverse, el otro el nivel de batería que tendrá el robot tras ejecutar la acción.
	\item Como precondiciones, se ha añadido \newlispinline/(batterylevel ?r ?bbefore)/, especifica que el nivel de batería del robot antes de ejecutar la acción debe ser \newlispinline/?bbefore/. \newlispinline/(cambio ?bbefore ?bafter)/ cambiará el nivel de batería.
	\item Como efecto, \newlispinline/(not (batterylevel ?r ?bbefore))/ establece que el nivel de batería del robot ya no es el mismo que tenía antes de ejecutar la acción, y \newlispinline/(batterylevel ?r ?bafter)/ establece el nivel actual a \newlispinline/?bafter/
\end{itemize}

\lispscript{../d2.1.pddl}{Dominio2.pddl}

\subsection{Problema}
\label{sub:Problema}

El problema definido para éste dominio consiste en 2 robots, 10 paquetes y 4 habitaciones. Los niveles de batería para los dos robots es \newlispinline/fl2/.

\lispscript{../p1e2.pddl}{p1e2.pddl}

El plan generado:

\begin{bashcode}
	        0: CHARGE R2
	        1: CHARGE R1
	        2: MOVE R1 HAB0 HAB1
	        3: MOVE R2 HAB0 HAB1
	        4: MOVE R1 HAB1 HAB2
	        5: MOVE R2 HAB1 HAB0
	        6: MOVE R1 HAB2 HAB1
	        7: PICK P1 HAB0 R2
	        8: MOVE R2 HAB0 HAB1
	        9: MOVE R2 HAB1 HAB2
	       10: DROP P1 HAB2 R2
	       11: MOVE R1 HAB1 HAB2
	       12: PICK P6 HAB2 R2
	       13: MOVE R2 HAB2 HAB1
	       14: CHARGE R2
	       15: MOVE R2 HAB1 HAB0
	       16: DROP P6 HAB0 R2
	       17: MOVE R2 HAB0 HAB1
	       18: MOVE R2 HAB1 HAB0
	       19: PICK P2 HAB0 R2
	       20: MOVE R2 HAB0 HAB1
	       21: MOVE R2 HAB1 HAB2
	       22: DROP P2 HAB2 R2
	       23: PICK P7 HAB2 R1
	       24: MOVE R1 HAB2 HAB1
	       25: CHARGE R1
	       26: MOVE R1 HAB1 HAB0
	       27: DROP P7 HAB0 R1
	       28: PICK P3 HAB0 R1
	       29: MOVE R1 HAB0 HAB1
	       30: MOVE R1 HAB1 HAB2
	       31: DROP P3 HAB2 R1
	       32: PICK P8 HAB2 R1
	       33: MOVE R1 HAB2 HAB1
	       34: MOVE R1 HAB1 HAB0
	       35: CHARGE R1
	       36: DROP P8 HAB0 R1
	       37: PICK P4 HAB0 R1
	       38: MOVE R1 HAB0 HAB1
	       39: MOVE R1 HAB1 HAB2
	       40: DROP P4 HAB2 R1
	       41: PICK P9 HAB2 R1
	       42: MOVE R1 HAB2 HAB1
	       43: MOVE R1 HAB1 HAB0
	       44: DROP P9 HAB0 R1
	       45: PICK P5 HAB0 R1
	       46: MOVE R1 HAB0 HAB1
	       47: CHARGE R1
	       48: MOVE R1 HAB1 HAB2
	       49: DROP P5 HAB2 R1
	       50: PICK P10 HAB2 R1
	       51: MOVE R1 HAB2 HAB1
	       52: MOVE R1 HAB1 HAB0
	       53: DROP P10 HAB0 R1
	       54: MOVE R1 HAB0 HAB1
	       55: MOVE R1 HAB1 HAB2
\end{bashcode}

\section{Ejercicio 3}
\label{sec:Ejercicio 3}

\miit{Escribir un dominio de planificación en PDDL, modificando el dominio del anterior
ejercicio, de manera que se puedan utilizar ahora dos acciones diferentes, moverse
rápido y moverse lento tales que moverse rápido consume más unidades de fuel que
moverse lento. Probarlo con varios problemas.}

\subsection{Dominio}
\label{sub:Dominio}

Éste problema es una extensión del anterior, simplemente se ha añadido una acción \newlispinline/move-fast/ que consume más unidades de batería:

\begin{newlispcode}
	(:action move-fast
	  :parameters (?r - robot ?from ?to - habitacion ?bbefore ?bafter - flevel)
	  :precondition (and
	                  (at ?r ?from)
	                  (conectada ?from ?to)
	                  (batterylevel ?r ?bbefore)
	                  (cambio-fast ?bbefore ?bafter))

	  :effect (and
	            (at ?r ?to)
	            (not (at ?r ?from))
	            (not (batterylevel ?r ?bbefore) )
	            (batterylevel ?r ?bafter)
	            )
	)
\end{newlispcode}

Y un predicado \newlispinline/(cambio-fast ?n1 ?n2 - flevel)/.

\subsection{Problema}
\label{sub:Problema}

El problema tiene un robot, y los niveles de batería del 0 al 2.

\lispscript{../p0e3.pddl}{p0e3.pddl}

Y el plan generado:

\begin{bashcode}
	        0: PICK P1 HAB0 R1
	        1: MOVE-FAST R1 HAB0 HAB1 FL2 FL0
	        2: MOVE-FAST R1 HAB1 HAB2 FL0 FL2
	        3: DROP P1 HAB2 R1
	        4: PICK P2 HAB2 R1
	        5: MOVE R1 HAB2 HAB1 FL2 FL1
	        6: MOVE R1 HAB1 HAB0 FL1 FL0
	        7: DROP P2 HAB0 R1
	        8: MOVE-FAST R1 HAB0 HAB1 FL0 FL2
\end{bashcode}

\section{Ejercicio 4}
\label{sec:Ejercicio 4}

\miit{Escribir un dominio de planificación en PDDL 2.1 que responda a los requisitos
explicados en los anteriores ejercicios, utilizando las capacidades expresivas de PDDL
2.1, es decir, representando funciones numéricas. Al igual que en los anteriores
ejercicios, definir distintos problemas para comprobar que vuestro dominio es
correcto.}

\subsection{Dominio}
\label{sub:Dominio}

Éste dominio se ha especificado haciendo uso de funciones, (\newlispinline/:functions/). Para ello, se ha modificado el dominio1, añadiendo la siguiente función:

\begin{newlispcode}
	(:functions
		(battery-left ?r - robot)
	)
\end{newlispcode}

Que devolverá la cantidad de batería disponible.

Además, la acción \newlispinline/move/ se ha modificado para que modifique el nivel de batería de un robot, en su precondición se exige que el nivel de batería tenga un valor mínimo \newlispinline/(>= (battery-left ?r) 2)/, y como efecto, se decrementa el nivel de batería en un valor \newlispinline/(decrease (battery-left ?r) 2)/.

Cuando el nivel de batería no permite al robot moverse, éste puede recargar la batería con la acción \newlispinline/charge/:

\begin{newlispcode}
  (:action charge
    :parameters (?r - robot)
    :precondition (< (battery-left ?r ) 2)
    :effect (assign (battery-left ?r ) 10)
  )
\end{newlispcode}

Ésta acción requiere que el nivel de batería sea menor que 2, y asigna un valor de 10 a la carga.

El código completo del dominio es:

\lispscript{../d4.pddl}{d4.pddl}

\subsection{Problema}
\label{sub:Problema}

Para éste problema se han especificado tres robots, diez paquetes y cuatro habitaciones, cada robot con la batería cargada a 100 unidades:

\lispscript{../p0e4.pddl}{p0e4.pddl}

El plan trazado:

\begin{bashcode}
        0: PICK P9 HAB4 R3
        1: MOVE-FAST R3 HAB4 HAB3
        2: MOVE-FAST R1 HAB0 HAB1
        3: MOVE-FAST R2 HAB0 HAB1
        4: MOVE R1 HAB1 HAB2
        5: MOVE-FAST R3 HAB3 HAB1
        6: DROP P9 HAB1 R3
        7: MOVE-FAST R3 HAB1 HAB3
        8: MOVE-FAST R3 HAB3 HAB4
        9: PICK P9 HAB1 R2
       10: MOVE R2 HAB1 HAB0
       11: DROP P9 HAB0 R2
       12: MOVE-FAST R2 HAB0 HAB1
       13: MOVE R2 HAB1 HAB0
       14: MOVE R1 HAB2 HAB1
       15: PICK P1 HAB0 R2
       16: MOVE-FAST R2 HAB0 HAB1
       17: MOVE R2 HAB1 HAB2
       18: DROP P1 HAB2 R2
       19: MOVE R1 HAB1 HAB2
       20: PICK P6 HAB2 R2
       21: MOVE R2 HAB2 HAB1
       22: MOVE R2 HAB1 HAB0
       23: DROP P6 HAB0 R2
       24: MOVE-FAST R2 HAB0 HAB1
       25: MOVE R2 HAB1 HAB0
       26: MOVE R1 HAB2 HAB1
       27: PICK P2 HAB0 R2
       28: MOVE-FAST R2 HAB0 HAB1
       29: MOVE R2 HAB1 HAB2
       30: DROP P2 HAB2 R2
       31: MOVE R1 HAB1 HAB2
       32: PICK P7 HAB2 R2
       33: MOVE R2 HAB2 HAB1
       34: MOVE R2 HAB1 HAB0
       35: DROP P7 HAB0 R2
       36: MOVE-FAST R2 HAB0 HAB1
       37: MOVE R2 HAB1 HAB0
       38: PICK P3 HAB0 R2
       39: MOVE-FAST R2 HAB0 HAB1
       40: MOVE-FAST R2 HAB1 HAB2
       41: DROP P3 HAB2 R2
       42: PICK P8 HAB2 R1
       43: MOVE R1 HAB2 HAB1
       44: MOVE R1 HAB1 HAB0
       45: DROP P8 HAB0 R1
       46: PICK P4 HAB0 R1
       47: MOVE-FAST R1 HAB0 HAB1
       48: MOVE R1 HAB1 HAB2
       49: DROP P4 HAB2 R1
       50: PICK P10 HAB2 R1
       51: MOVE R1 HAB2 HAB1
       52: MOVE R1 HAB1 HAB0
       53: DROP P10 HAB0 R1
       54: PICK P5 HAB0 R1
       55: MOVE-FAST R1 HAB0 HAB1
       56: MOVE R1 HAB1 HAB2
       57: DROP P5 HAB2 R1
\end{bashcode}

\end{document}
