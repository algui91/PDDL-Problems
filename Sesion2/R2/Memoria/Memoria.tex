%%% LaTeX Template: Two column article
%%%
%%% Source: http://www.howtotex.com/
%%% Feel free to distribute this template, but please keep to referal to http://www.howtotex.com/ here.
%%% Date: February 2011

%%% Preamble
\documentclass[	DIV=calc,%
							paper=a4,%
							fontsize=11pt]{scrartcl}	 					% KOMA-article class

\usepackage[spanish]{babel}										% English language/hyphenation
\usepackage[protrusion=true,expansion=true]{microtype}				% Better typography
\usepackage{amsmath,amsfonts,amsthm}					% Math packages
%\usepackage[pdftex]{graphicx}									% Enable pdflatex
\usepackage[svgnames]{xcolor}									% Enabling colors by their 'svgnames'
\usepackage[hang, small,labelfont=bf,up,textfont=it,up]{caption}	% Custom captions under/above floats
\usepackage{epstopdf}												% Converts .eps to .pdf
\usepackage{subfig}													% Subfigures
\usepackage{booktabs}												% Nicer tables
\usepackage{fix-cm}													% Custom fontsizes
\usepackage{hyperref}

\usepackage[T1]{fontenc}
\usepackage[utf8]{inputenc}
\usepackage[light,math]{kurier}


%%% Custom sectioning (sectsty package)
\usepackage{sectsty}													% Custom sectioning (see below)
\allsectionsfont{%															% Change font of al section commands
	\usefont{T1}{mdugm}{b}{it}%										% bch-b-n: CharterBT-Bold font
	}

\sectionfont{%																% Change font of \section command
	\usefont{T1}{mdugm}{b}{it}%										% bch-b-n: CharterBT-Bold font
	}

%%% Headers and footers
\usepackage{fancyhdr}												% Needed to define custom headers/footers
	\pagestyle{fancy}														% Enabling the custom headers/footers
\usepackage{lastpage}

% Header (empty)
\lhead{}
\chead{}
\rhead{}
% Footer (you may change this to your own needs)
\lfoot{\footnotesize \miit{Alejandro Alcalde Barros} \textbullet ~}
\cfoot{}
\rfoot{\footnotesize Página \thepage\ de \pageref{LastPage}}	% "Page 1 of 2"
\renewcommand{\headrulewidth}{0.0pt}
\renewcommand{\footrulewidth}{0.4pt}



%%% Creating an initial of the very first character of the content
\usepackage{lettrine}
\newcommand{\initial}[1]{%
     \lettrine[lines=3,lhang=0.3,nindent=0em]{
     				\color{DarkGoldenrod}
     				{\textsf{#1}}}{}}



%%% Title, author and date metadata
\usepackage{titling}															% For custom titles

\newcommand{\HorRule}{\color{DarkGoldenrod}%			% Creating a horizontal rule
									  	\rule{\linewidth}{1pt}%
										}

\pretitle{\vspace{-30pt} \begin{flushleft} \HorRule
				\fontsize{50}{50} \usefont{T1}{kurier}{l}{it} \color{DarkRed} \selectfont
				}
\title{Memoria Relación de Ejercicios PDDL}					% Title of your article goes here
\posttitle{\par\end{flushleft}\vskip 0.5em}

\preauthor{\begin{flushleft}
					\large \lineskip 0.5em \usefont{T1}{mdugm}{m}{it} \color{DarkRed}}
\author{Alejandro Alcalde, }											% Author name goes here
\postauthor{\footnotesize \usefont{OT1}{mdugm}{m}{it} \color{Black}
					Universidad de Granada 								% Institution of author
					\par\end{flushleft}\HorRule}

\date{\usefont{T1}{mdugm}{b}{it}\selectfont\today}																				% No date

\newcommand{\miit}[1]{{\usefont{T1}{mdugm}{m}{it}\selectfont #1}}

\usepackage{minted}

\newminted{bash}{
		numbersep=5pt,
		autogobble=true,
		frame=lines,
		framesep=2mm,
		fontsize=\scriptsize,
		tabsize=2,
		fontfamily=DejaVuSansMono-TLF,
}

\newminted{newlisp}{
		numbersep=5pt,
		autogobble=true,
		frame=lines,
		framesep=2mm,
		fontsize=\scriptsize,
		tabsize=2,
		fontfamily=DejaVuSansMono-TLF,
}

\newmintedfile[myLisp]{newlisp}{
    linenos,
    numbersep=5pt,
		autogobble=true,
    frame=lines,
    framesep=2mm,
    fontsize=\scriptsize,
		tabsize=2,
		fontfamily=DejaVuSansMono-TLF,
}

\newcommand{\lispscript}[2]{
    \myLisp[label=#2]{#1}
}

\newmintinline{newlisp}{fontsize=\scriptsize, fontfamily=DejaVuSansMono-TLF}

%%% Begin document
\begin{document}
\maketitle
\thispagestyle{fancy} 			% Enabling the custom headers/footers for the first page
% The first character should be within \initial{}
\textbf{En ésta memoria se describe cómo se han planteado, y resuelto, los distintos problemas propuestos sobre PDDL en la asignatura Técnicas de los Sisteas Inteligentes.}

\section{Ejercicio 1}

\miit{El siguiente problema a resolver (problema-zeno-V01.pddl) consiste en transportar 3 personas
(inicialmente en las ciudades C1, C2 y C3) a la ciudad C5, considerando que el avión está en la ciudad C4.
Se asume al igual que en el problema ejemplo que no hay restricciones de fuel.}\\

La razón por la cual el dominio \newlispinline/problema-zeno-V01.pddl/ no resolvía el problema 1 residía en que no se había contemplado
la posibilidad de que el avión estuviera inicialmente en una ciudad sin pasajeros. Es decir, la persona está en la ciudad de origen,
pero el avión está en otra ciudad, distinta a la del pasajero y a la de destino.

Se ha añadido un método a la tarea \newlispinline/transport-person/ para controlar esta situación:

\begin{newlispcode}
	;; La persona está en la ciudad de origen, pero el avión está en otra ciudad,
	;; distinta a ?to y a ?from, la llamamos ?c2
	(:method Case3
		:precondition (
			and
				(at ?p - person ?from - city)
				(at ?a - aircraft ?c2 - city)
		)
		:tasks (
				;; Mueve el avión de donde está a la ciudad origen de la persona
				(mover-avion ?a ?c2 ?from)
				(board ?p ?a ?from)
				(mover-avion ?a ?from ?to)
				(debark ?p ?a ?to )
		)
	)
\end{newlispcode}

\subsection{Plan trazado}
\label{sub:Plan trazado}

\begin{bashcode}
	:action (fly a1 c4 c1) start: 05/06/2007 08:00:00 end: 05/06/2007 23:00:00
	:action (board p1 a1 c1) start: 05/06/2007 23:00:00 end: 06/06/2007 00:00:00
	:action (fly a1 c1 c5) start: 06/06/2007 00:00:00 end: 06/06/2007 10:00:00
	:action (debark p1 a1 c5) start: 06/06/2007 10:00:00 end: 06/06/2007 11:00:00
	:action (fly a1 c5 c2) start: 06/06/2007 11:00:00 end: 07/06/2007 02:00:00
	:action (board p2 a1 c2) start: 07/06/2007 02:00:00 end: 07/06/2007 03:00:00
	:action (fly a1 c2 c5) start: 07/06/2007 03:00:00 end: 07/06/2007 18:00:00
	:action (debark p2 a1 c5) start: 07/06/2007 18:00:00 end: 07/06/2007 19:00:00
	:action (fly a1 c5 c3) start: 07/06/2007 19:00:00 end: 08/06/2007 10:00:00
	:action (board p3 a1 c3) start: 08/06/2007 10:00:00 end: 08/06/2007 11:00:00
	:action (fly a1 c3 c5) start: 08/06/2007 11:00:00 end: 09/06/2007 02:00:00
	:action (debark p3 a1 c5) start: 09/06/2007 02:00:00 end: 09/06/2007 03:00:00
\end{bashcode}


\section{Ejercicio 2}
\label{sec:Ejercicio 2}

\miit{El problema 2 (fichero problema-zeno-V02.pddl) consiste en asumir que hay restricciones de fuel. El fuel
inicial del avión es de 200 y la capacidad total de 300. Deben contemplarse ahora acciones de repostaje.
La situación de partida de personas y avión es la misma que en el problema anterior.}\\

Esta vez bastaba con añadir un nuevo método en la tarea \newlispinline/mover-avion/, dicha tarea permite al avión repostar combustible en la ciudad origen de no tener suficiente para volar al destino:
\begin{newlispcode}
  (:method fuel-insufuciente
    ;; No hay combustible para ir de ?from a ?to en el avión ?a
    :precondition ( not (hay-fuel ?a ?from ?to ))
    :tasks (
      (refuel ?a ?from)
      (fly ?a ?from ?to)
    )
  )
\end{newlispcode}

\subsection{Plan trazado}
\label{sub:Plan trazado}

\begin{bashcode}
	:action (fly a1 c4 c1) start: 05/06/2007 08:00:00 end: 05/06/2007 23:00:00
	:action (board p1 a1 c1) start: 05/06/2007 23:00:00 end: 06/06/2007 00:00:00
	:action (refuel a1 c1) start: 06/06/2007 00:00:00 end: 16/06/2007 10:00:00
	:action (fly a1 c1 c5) start: 16/06/2007 10:00:00 end: 16/06/2007 20:00:00
	:action (debark p1 a1 c5) start: 16/06/2007 20:00:00 end: 16/06/2007 21:00:00
	:action (fly a1 c5 c2) start: 16/06/2007 21:00:00 end: 17/06/2007 12:00:00
	:action (board p2 a1 c2) start: 17/06/2007 12:00:00 end: 17/06/2007 13:00:00
	:action (refuel a1 c2) start: 17/06/2007 13:00:00 end: 27/06/2007 23:00:00
	:action (fly a1 c2 c5) start: 27/06/2007 23:00:00 end: 28/06/2007 14:00:00
	:action (debark p2 a1 c5) start: 28/06/2007 14:00:00 end: 28/06/2007 15:00:00
	:action (fly a1 c5 c3) start: 28/06/2007 15:00:00 end: 29/06/2007 06:00:00
	:action (board p3 a1 c3) start: 29/06/2007 06:00:00 end: 29/06/2007 07:00:00
	:action (refuel a1 c3) start: 29/06/2007 07:00:00 end: 11/07/2007 19:00:00
	:action (fly a1 c3 c5) start: 11/07/2007 19:00:00 end: 12/07/2007 10:00:00
	:action (debark p3 a1 c5) start: 12/07/2007 10:00:00 end: 12/07/2007 11:00:00
\end{bashcode}

\section{Ejercicio 3}
\label{sec:Ejercicio 3}

\miit{Este problema (fichero problema-zeno-V03.pddl) consiste en considerar acciones de vuelo lento y
rápido para tratar de transportar las personas lo más rápido posible con un límite de fuel. En el dominio
se tienen que codificar los métodos y tareas de manera que se priorice el uso de acciones de velocidad
rápida. El límite de fuel se define con la función (fuel-limit) y es asignado en el estado inicial a 1500. La
suma total de fuel gastado en todos los transportes no puede superar 1500 unidades, pero el avión debe
viajar siempre lo más rápido posible.}\\

El código de ambos problemas anteriores se ha reutilizado en éste. Además, se ha añadido lo siguiente:

\begin{itemize}
	\item Se ha definido un límite en el gasto total de combustible, \newlispinline/(fuel-limit)/.
	\item Ahora existen dos métodos de repostaje:
	\begin{itemize}
		\item \newlispinline/fuel-insufuciente-vuelo-rapido/, trata la casuística en la que no hay combustible suficiente para desplazarse a una velocidad rápida. Para ello, el límite de combustible debe ser menor que la cantidad usada hasta el momento más el coste de moverse de una ciudad a otra a velocidad alta
		\item \newlispinline/fuel-insufuciente-/, trata el caso en el que no haya fuel suficiente para ir de una ciudad a otra, a velocidad lenta.
	\end{itemize}
	\item y dos modos de vuelo:
	\begin{itemize}
		\item \newlispinline/vuelo-rapido/, El fuel del avión debe ser mayor del que va a consumir al ir de una ciudad a otra a velocidad rápida, y el total usado a velocidad rápida no debe superar el límite.
		\item \newlispinline/vuelo-lento/, debe haber combustible para ir de una ciudad a otra, y el total consumido no debe superar el límite.
	\end{itemize}
\end{itemize}

A continuación se muestran los métodos

\begin{newlispcode}
;; Si no hay combustible
(:method fuel-insufuciente
	;; No hay combustible para ir de ?from a ?to en el avion ?a
	:precondition ( not (hay-fuel ?a ?from ?to ))
	:tasks (
		(refuel ?a ?from)
		(fly ?a ?from ?to)
	)
)
(:method fuel-insufuciente-vuelo-rapido
	:precondition (
			and
			(not (hay-fuel ?a ?from ?to))
			(< (+ (+ (total-fuel-used)
							(* (distance ?from ?to) (fast-burn ?a)) )
						(capacity ?a)) (fuel-limit))
	)
	:tasks (
			(refuel ?a ?from)
			(zoom ?a ?from ?to)
	)
)
;; Igual para el lento
(:method vuelo-lento
	:precondition (
			and
			(hay-fuel ?a ?from ?to)
			(<  (+ (total-fuel-used)
					(* (distance ?from ?to) (slow-burn ?a)))
				(fuel-limit))
	)
	:tasks (
			(fly ?a ?from ?to)
	)
)
;; Hay combustible para ir rápido y el fuel gastado no supera el límite
(:method vuelo-rapido
	:precondition (
			and
			(>= (fuel ?a)
				(*  (distance ?from ?to)
						(fast-burn ?a)))
			(< (+ (total-fuel-used)
				(* (distance ?from ?to) (fast-burn ?a)))
			(fuel-limit))
	)
	:tasks (
			(zoom ?a ?from ?to)
	)
)
\end{newlispcode}

\subsection{Plan trazado}
\label{sub:Plan trazado}

\begin{bashcode}
	:action (fly a1 c4 c1) start: 05/06/2007 08:00:00 end: 05/06/2007 23:00:00
	:action (board p1 a1 c1) start: 05/06/2007 23:00:00 end: 06/06/2007 00:00:00
	:action (refuel a1 c1) start: 06/06/2007 00:00:00 end: 16/06/2007 10:00:00
	:action (zoom a1 c1 c5) start: 16/06/2007 10:00:00 end: 16/06/2007 15:00:00
	:action (debark p1 a1 c5) start: 16/06/2007 15:00:00 end: 16/06/2007 16:00:00
	:action (refuel a1 c5) start: 16/06/2007 16:00:00 end: 25/06/2007 00:00:00
	:action (zoom a1 c5 c2) start: 25/06/2007 00:00:00 end: 25/06/2007 08:00:00
	:action (board p2 a1 c2) start: 25/06/2007 08:00:00 end: 25/06/2007 09:00:00
	:action (refuel a1 c2) start: 25/06/2007 09:00:00 end: 07/07/2007 21:00:00
	:action (zoom a1 c2 c5) start: 07/07/2007 21:00:00 end: 08/07/2007 05:00:00
	:action (debark p2 a1 c5) start: 08/07/2007 05:00:00 end: 08/07/2007 06:00:00
	:action (refuel a1 c5) start: 08/07/2007 06:00:00 end: 20/07/2007 18:00:00
	:action (fly a1 c5 c3) start: 20/07/2007 18:00:00 end: 21/07/2007 09:00:00
	:action (board p3 a1 c3) start: 21/07/2007 09:00:00 end: 21/07/2007 10:00:00
	:action (fly a1 c3 c5) start: 21/07/2007 10:00:00 end: 22/07/2007 01:00:00
	:action (debark p3 a1 c5) start: 22/07/2007 01:00:00 end: 22/07/2007 02:00:00
\end{bashcode}
\end{document}
